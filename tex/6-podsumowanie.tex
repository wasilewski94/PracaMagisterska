\newpage % Rozdziały zaczynamy od nowej strony.
\cleardoublepage % Zaczynamy od nieparzystej strony
\pagestyle{headings}

\section{Podsumowanie}

Implementacja Sztucznych Sieci Neuronowych w układach FPGA daje wyraźne przyspieszenie czasu wykonywania obliczeń w stosunku do rozwiązań programowych. Główne założenia projektu zostały spełnione, dokonano porównania wydajności implementacji Sztucznej Sieci Neuronowej w układzie FPGA przy użyciu syntezy wysokiego poziomu do rozwiązania programowego uruchamianego na komputerze PC. W celu osiągnięcia dużego współczynnika akceleracji obliczeń wykorzystano metody optymalizacji dostępne w narzędziu Vivado HLS. Dokonano przeglądu metod umożliwiających zmniejszenie opóźnień wykonywanych operacji oraz ograniczenie zużycia zasobów układu FPGA.

Podczas projektu znaleziono elementy implementacji wprowadzające największe opóźnienia i z powodzeniem wyeliminowano lub zmniejszono ich wpływ na czas wykonania algorytmu oraz zużycie zasobów sprzętowych. W trakcie tworzenia implementacji napotkano kilka problemów, utrudniających optymalizację w narzędziu Vivado HLS. Ostatecznie osiągnięto rozwiązanie zapewniające akcelerację obliczeń w Sztucznych Sieciach Neuronowych. 

% coś jeszcsze o najlepszym wyniku jaki wyszedł i ograniczeniach i jak się zmieniały zasoby

\subsection{Wnioski dotyczące techniki HLS}
W dziedzinie rozwijania oprogramowania można zaobserwować tendencję przechodzenia w kierunku rozwiązań wysokopoziomowych. Ta zależność jest widoczna również w przypadku programowania układów FPGA. Powstają nowe biblioteki i narzędzia usprawniające proces tworzenia implementacji akceleratorów obliczeń. W większości przypadków programista nie musi już posiadać dużej wiedzy o sprzęcie, aby osiągnąć rozwiązanie na akceptowalnym poziomie. Jednak w przypadku złożonych algorytmów bywa, że osiągnięcie implementacji optymalnej, porównywalnej z implementacją w języku HDL jest trudne lub nawet niemożliwe.

Technika HLS okazała się odpowiednim narzędziem do wykonania implementacji Sztucznych Sieci Neuronowych. Osiągnięto zadowalające rezultaty w stosunkowo niedługim czasie, a możliwość implementowania algorytmu w języku C++, ułatwiła wprowadzanie zmian w projekcie. Dzięki odpowiednim dyrektywom i parametrom można było odpowiednio zoptymalizować dane rozwiązanie. Ustawiając odpowiednie parametry, można osiągnąć kompromis pomiędzy zużyciem zasobów a szybkością wykonywania obliczeń w algorytmie ANN. Problematyczny w procesie optymalizacji był czas wykonywania syntezy w narzędziu Vivado HLS oraz implementacji w środowisku Vivado, który w~niektórych przypadkach wynosił kilka godzin. 

\subsection{Możliwości rozwoju projektu}

Użycie w projekcie metody HLS umożliwiło znalezienie elementów implementacji wprowadzających duże opóźnienia i powodujących wykorzystanie dużej ilości zasobów układu FPGA. W końcowym etapie projektu problematyczny okazał się długi czas syntezy i implementacji. Projekt można rozwinąć, stosując bardziej skomplikowane architektury Sztucznych Sieci Neuronowych. Z powodu ograniczeń czasu projektu nie udało się zaimplementować warstwy splotowej, która umożliwia zmniejszenie liczby parametrów sieci przy jednoczesnym osiągnięciu wysokiej dokładności klasyfikacji obrazów. Dodatkowym usprawnieniem może być zastosowanie bardziej wydajnego interfejsu komunikacji akceleratora HLS z procesorem w układzie Zynq, zapewniającego wydajniejszy dostęp do pamięci oraz obrazu z kamery w czasie rzeczywistym.

W pracy zaprezentowano zastosowanie Sztucznej Sieci Neuronowej do rozpoznawania obiektów 
znajdujących się na obrazie z kamery podłączanej przez USB w czasie rzeczywistym. Część algorytmu 
odpowiedzialna za detekcję cyfr na obrazie była uruchamiana na procesorze ARM przy użyciu 
biblioteki OpenCV, co wprowadzało spore opóźnienia i~ograniczało wartość FPS oraz maksymalną liczbę 
cyfr możliwych do znalezienia w danym momencie. W przypadku zastosowania systemu do zadań 
klasyfikacji obrazów możliwym usprawnieniem projektu jest zastosowanie kamery podłączanej przez 
interfejs równoległy bezpośrednio do układu FPGA w celu zminimalizowania opóźnień. Dodatkowym 
usprawnieniem może być dalsza optymalizacja algorytmu przy użyciu techniki HLS w~celu osiągnięcia maksymalnej akceleracji obliczeń i minimalnego zużycia zasobów logiki programowalnej.

Zaletą rozwiązania zawartego w projekcie jest duża uniwersalność. Dzięki możliwości zmiany 
parametrów sieci w funkcji akceleratora użytkownik ma możliwość dokonywania zmian w modelu ANN z 
poziomu aplikacji. Daje to duże możliwości zastosowania systemu w różnych zadaniach związanych z 
dziedziną \emph{Machine Learning}. Zastosowanie układu FPGA umożliwia znalezienie optymalnego 
rozwiązania zapewniającego kompromis pomiędzy zużyciem mocy a szybkością działania systemu, co jest bardzo istotne w przypadku zastosowania w systemach wbudowanych.
