\newpage % Rozdziały zaczynamy od nowej strony.
\cleardoublepage % Zaczynamy od nieparzystej strony
\pagestyle{headings}

\section{Podsumowanie}

Sztuczne Sieci Neuronowe są algorytmem, który bardzo dobrze wpasowuje się w zastosowanie układów FPGA. Akceleracja obliczeń w stosunku do rozwiązań programowych jest możliwa do osiągnięcia. Główne założenia projektu zostały spełnione, stworzono implementację Sztucznej Sieci Neuronowej w układzie FPGA przy użyciu syntezy wysokiego poziomu. Podczas projektu znaleziono elementy implementacji wprowadzające największe opóźnienia i z powodzeniem wyeliminowano lub zmniejszono ich wpływ na czas wykonania algorytmu oraz zużycie zasobów sprzętowych.

\subsection{Wnioski dotyczące techniki HLS}
W dziedzinie rozwijania oprogramowania można zaobserwować tendencję przechodzenia w kierunku rozwiązań wysokopoziomowych.
Ta zależność jest widoczna również w przypadku programowania układów FPGA. Powstają nowe biblioteki i narzędzia usprawniające proces tworzenia implementacji akceleratorów obliczeń. W większości przypadków programista nie musi już posiadać dużej wiedzy o sprzęcie, aby osiągnąć rozwiązanie na akceptowalnym poziomie. Jednak w przypadku złożonych algorytmów bywa, że osiągnięcie implementacji optymalnej, porównywalnej z implementacją w języku HDL jest bardzo trudne lub nawet niemożliwe.  

\subsection{Możliwości rozwoju projektu}
