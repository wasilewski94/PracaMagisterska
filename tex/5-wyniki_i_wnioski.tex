\newpage % Rozdziały zaczynamy od nowej strony.
\cleardoublepage % Zaczynamy od nieparzystej strony
\pagestyle{headings}

\section{Wyniki i wnioski}

Celem pracy było zaprojektowanie, nauczenie i przetestowanie działania algorytmu Sztucznej Sieci Neurnonowej z użyciem układu FPGA oraz porównanie z rozwiązaniem programowym. Podczas projektu powstało kilka modeli Sztucznej Sieci Neuronowej klasyfikującej odręcznie pisane cyfry. Aby porównać rozwiązanie, realizowane w technice HLS z implementacją przy użyciu pakietu \emph{keras}, każdy z modeli poddano testom, które zostały podzielone na dwie części:
\bigskip
\begin{enumerate}
  \item Uruchomienie sieci przy użyciu zbioru testowego 10000 cyfr z bazy MNIST, w celu oszacowania dokładności i szybkości działania algorytmu.
  \item Test wykonany w czasie rzeczywistym przy użyciu modułu kamery.
\end{enumerate}

Wyniki przeprowadzonych testów zostały zestawione w dalszej części rozdziału.

\subsection{Test modelu sieci z jedną warstwą ukrytą}


% Algorytmy rozpoznawania obiektów mogą być wywoływane na różne sposoby. 
% Jedną z metod jest rejestrowanie obrazu z możliwie maksymalną ilością klatek 
% na sekundę, analizowanie każdej ramki, wyszukiwanie obiektów i klasyfikacja za 
% pomocą algorytmu ANN. Drugim, prostszym w implementacji sposobem, jest wywoływanie 
% zarejestrowania obrazu w momencie, gdy użytkownik, chce dokonać klasyfikacji 
% obiektu, który znajduje się w zasięgu obiektywu kamery, a na zarejestrowanym 
% obrazie nie ma innych obiektów. Z powodu ograniczeń zasobów systemu, na którym 
% aplikacja była testowana oraz ograniczonego czasu wykonania projektu, podjęto 
% decyzję o zastosowaniu drugiej metody. 


