\newpage % Rozdziały zaczynamy od nowej strony.
\cleardoublepage % Zaczynamy od nieparzystej strony
\pagestyle{headings}

\section{Wybór sprzętu}

Dlaczego Sztuczne Sieci Neuronowe odpala się na GPU?
Dlaczego nie CPU i GPU tylko FPGA?
W większości przypadków Sztuczne Sieci Neuronowe są 
Typically, neural networks are designed, trained, and executed on a conventional processor, often with GPU acceleration. But for embedded devices which may need to process data at multiple-MHz sample rates, the computational requirements can be overwhelming for an embedded processor where no GPU is available, creating a tempting opportunity for FPGA acceleration. (https://github.com/Xilinx/RFNoC-HLS-NeuralNet)
Co z ASIC, dlaczego rzadko się je stosuje, jak wygląda proces tworzenia?
Dlaczego PC ma swoje ograniczenia? Jakie możliwości mają nowe algorytmy uruchamiane na PC?
Jaką przewagę dają układy FPGA, skąd się to bierze? porównanie zużycia mocy itp..\\

Trzeba tu tylko uważać, żeby nie powielać tekstu z rozdziału cel i zakres pracy.


\begin{table}[h] \centering
  \caption{Porównanie cen płytek z układami Zynq firmy Xilinx}
  \centering
  \begin{tabular} {c|c|c} \hline \label{tab:ceny}
      Nazwa płytki & Układ SoC & Cena \\ \hline
      Z-turn Board MYS-7Z010-C-S & XC7Z010-1CLG400C & 99\$\tablefootnote{http://www.myirtech.com/list.asp?id=502} \\ 
      Z-turn Board MYS-7Z020-C-S & XC7Z020-1CLG400C  & 119\$\footnotemark[1] \\
      Zybo Z7-10 Development Board & XC7Z010-1CLG400C & 199\$\tablefootnote{https://store.digilentinc.com/zybo-z7-zynq-7000-arm-fpga-soc-development-board/} \\
      Zybo Z7-20 Development Board & XC7Z020-1CLG400C & 299\$\footnotemark[2] \\
      ZedBoard Zynq-7000 & XC7Z020-CLG484-1 & 449\$\tablefootnote{https://store.digilentinc.com/zedboard-zynq-7000-arm-fpga-soc-development-board/} \\
  \end{tabular}
\end{table}


\subsection{Z-turn Board}

Z-turn Board (Rys. \ref{zturn_board} jest komputerem jednopłytkowym 
(ang. SBC – \emph{Single Board Computer}), opartym o układ SoC Xilinx 
Zynq-7020 (XC7Z020-1CLG400C), zawierającym dwurdzeniowy procesor ARM Cortex-A9 
i układ FPGA Artix 7. Producentem płytki jest firma MYIR Tech Limited (ang. \emph{Make 
Your Ideas Real} ), dostarczająca sprzęt bazujący na procesorach ARM oraz 
oprogramowanie do swoich produktów\cite{myir}. 

\begin{figure}[h]
  \centering
  \includegraphics[width=0.5\textwidth]{img/zturn_board.jpg}
  \caption{Płytka Z-turn-Board 7020}
  \label{zturn_board}
\end{figure}

Biorąc pod uwagę parametry, płytka 
charakteryzuje się wysokim stosunkiem ceny do jakości, podstawowa wersja kosztuje
99\$. Dla porównania płytka Zybo Z7-20 kosztuje 199\$. Zestawienie cen płytek 
zawierających układ Zynq XC7Z010 oraz XC7Z020 znajduje się w Tabeli \ref{tab:ceny}. 

\subsubsection{Interfejsy komunikacji}

Płytka Z-turn posiada interfejsy UART oraz Ethernet, które zostały wykorzystane do komunikacji komputera PC z systemem przy użyciu portu szeregowego  
i protokołu SSH (ang. \emph{Secure Shell}). Istnieje również możliwość podłączenia wyświetlacza bezpośrednio do płytki przy użyciu portu HDMI oraz innych peryferiów przy użyciu portu USB. Dodatkowo producent oferuje płytkę rozszerzeniową Z-turn IO-Cape (Rys. \ref{iocape}), która zawiera porty do podłączenia kamery przez protokół DVP (ang. \emph{Digital Video Port}) oraz wyświetlacza LCD. 

\begin{figure}[h]
  \centering
  \includegraphics[width=0.5\textwidth]{img/iocape.png}
  \caption{Płytka rozszerzeniowa Z-turn IO Cape}
  \label{iocape}
\end{figure}

\subsection{Kamera}

Aby przetestować działanie systemu w czasie rzeczywistym, wykorzystano zewnętrzny 
moduł kamery. Przy wyborze sprzętu istotna była ceną modułu, interfejs komunikacji
oraz kompatybilność z SBC Z-turn Board. W przypadku wykorzystania kamery w czasie 
rzeczywistym bardzo ważne jest niskie opóźnienie w wysyłaniu kolejnych ramek obrazu i duża szybkość transferu obrazu. 

Producent płytki Z-turn Board oferuje kilka modułów kamer. Wśród nich znajdują się dwa moduły, które zostały przetestowane: MY-CAM002U USB Digital Camera Module (Rys.\ref{cam-usb}) oraz MY-CAM011B BUS Camera Module. (Rys. \ref{cam-dvp}). W Tabeli \ref{tab:kamery} przedstawiono porównanie testowanych modułów kamer. 

\begin{table}[h] \centering
  \caption{Porównanie testowanych modułów kamer}
  \centering
  \begin{tabular} {c|c|c} \hline \label{tab:kamery}
      & MY-CAM011B &  MY-CAM002U \\ \hline
      Maksymalna rozdzielczość & 1600x1200 pikseli & 1280x800 pikseli \\ \hline
      Pobór mocy w stanie aktywnym & 224 mW & 110 mW\\ \hline
      Format wyjścia & 8/10-bit RAW RGB & 10-bit RAW RGB \\
      & YUV422/YCbCr422 & \\
      & RGB565/555 & \\
      & GRB422 & \\ \hline
      Maksymalny transfer obrazu & 15 fps (1600x1200)  & 30 fps (1280x800) \\
      & 30 fps (800x600)  & 60 fps (640x480) \\
      & 30 fps (1280x720) & 30 fps (1280x720) \\
      & 24 fps (1366x768) & \\ \hline
      Cena modułu & 25\$\tablefootnote{http://www.myirtech.com/list.asp?id=534} & 19\$\tablefootnote{http://www.myirtech.com/list.asp?id=462} \\
    \end{tabular}
  \end{table}
  
  \begin{figure}[!h]
      \centering
      \includegraphics[width=0.6\textwidth]{img/my-cam011b.png}
      \caption{Moduł kamery MY-CAM011B BUS Camera Module}
      \label{cam-dvp}
    \end{figure}

Kamera MY-CAM011B zapewnia większą rozdzielczość maksymalną i posiada interfejs równoległy DVP. Skutkuje to mniejszymi opóźnieniami w wysyłaniu kolejnych ramek obrazu niż w przypadku kamery MY-CAM002U, podłączanej przez USB. 

Pomimo znajdującego się na płytce IO-Cape portu interfejsu DVP, moduł kamery MY-CAM011B niestety okazał się niekompatybilny z płytką Z-turn Board. Na schemacie płytki rozszerzeniowej IO-Cape widać (Rys. \ref{cam-schematic}), że sygnał zegara (CAM\_XCLK) nie jest dołączony do żadnego portu płytki Z-Turn Board. Piny CAM\_PWRDN i CAM\_RST również nie są dołączone do złącza J2 od strony płytki Z-turn Board. 

\begin{figure}[!h]
  \centering
  \includegraphics[width=0.7\textwidth]{img/cam-schematic.png}
  \caption{Schemat portu DVP na płytce rozszerzeniowej IO-Cape}
  \label{cam-schematic}
\end{figure}

Dołączenie samego sygnału zegara do konektora FPC byłoby problematyczne, więc zdecydowano się na zastosowanie adaptera\footnote{https://kamami.pl/zlacza-ffc-fpc-zif/579387-adapter-zlacza-fpcffc-05mm-30-pin-na-dip.html} złącza FPC/FFC o rastrze 0,5 mm na otwory DIP raster 2,54 mm. Dzięki temu możliwe było podłączenie modułu kamery przy użyciu przewodów do złącza J3 płytki rozszerzeniowej IO-Cape \cite{ZturnIOCapeSchematic}.


\begin{figure}[!h]
  \centering
  \includegraphics[width=0.7\textwidth]{img/dvp-adapter.jpg}
  \caption{Adapter złącza FPC/FFC o rastrze 0,5 mm na otwory DIP}
  \label{cam-schematic}
\end{figure}


Po podłączeniu kamery, zainstalowaniu odpowiednich sterowników i \emph{device-tree} oraz przeskanowaniu magistrali $I^2C$ przy użyciu narzędzia \emph{i2c-tools}, moduł kamery zwraca poprawnie swój adres. Jednak sterownik sensora kamery \emph{ov2659.c} nie rozpoznaje podłączonego urządzenia. Z uwagi na ograniczony czas projektu zdecydowano się na użycie modułu kamery MY-CAM002U USB Digital Camera Module (Rys.\ref{cam-usb}). Moduł umożliwia rejestrowanie obrazu w 30 klatkach na sekundę przy rozdzielczości 1280x640 pikseli i jest tańszy od modułu MY-CAM011B. Główną zaletą tego modułu jest łatwość podłączenia sprzętu zarówno do płytki Z-turn Board, jak i do komputera PC, co w znacznym stopniu przyspieszyło debugowanie. Kamera po podłączeniu do komputera PC działała, bez wprowadzania jakichkolwiek zmian w systemie, zarówno w dystrybucji Ubuntu, jak i w systemie Windows. Po dodaniu odpowiednich sterowników do jądra i uruchomieniu systemu Petalinux na płytce Z-turn Board, w systemie pojawiły się pliki urządzenia pod nazwą \emph{/dev/video0} oraz \emph{/dev/video1}, co umożliwiło odebranie obrazu przy pomocy interfejsu V4L2 (ang. \emph{Video for Linux 2}) oraz OpenCV.


\begin{figure}[!h]
  \centering
  \includegraphics[width=0.7\textwidth]{img/MY-CAM001U.jpg}
  \caption{Moduł kamery MY-CAM002U USB Digital Camera Module}
  \label{cam-usb}
\end{figure}


\subsection{Opcje bootowania systemu}

