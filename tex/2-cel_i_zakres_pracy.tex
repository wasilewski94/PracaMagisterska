\newpage % Rozdziały zaczynamy od nowej strony.
\cleardoublepage % Zaczynamy od nieparzystej strony
\pagestyle{headings}

\section{Cel i zakres pracy}
Celem pracy było zaprojektowanie i implementacja sztucznej sieci neuronowej 
przy wykorzystaniu systemu SoC (ang. System on Chip) i techniki HLS. 
Użycie metody HLS pozwala na projektowanie  przy wykorzystaniu języka 
C, C++ lub System C, co przyspiesza pracę nad projektem. Dodatkowo HLS umożliwia
korzystanie z wielu bibliotek, które pozwalają wygodnie używać funkcji,
które są wykorzystywane w implementacji Sztucznych Sieci Neuronowych. 
Efektem pracy powinno być stworzenie akceleratora, umożliwiającego osiągnięcie 
wzrostu wydajności w stosunku do rozwiązań software’owych.

\subsection{Motywacja}
Sztuczne Sieci Neuronowe są związane z dużą ilością obliczeń, które mogą 
być wykonywane równolegle. Pozwala to osiągnąć krótszy czas wykonania 
programu, co ma duże znaczenie dla zastosowań w systemach działających 
w czasie rzeczywistym np. w branży Automotive. Aby osiągnąć przyspieszenie 
obliczeń stosuje się różne metody. Jednym z najpopularniejszych obecnie sposobów 
na zwiększenie wydajności algorytmów AI jest wykorzystanie kart graficznych GPU 
(ang. \emph{Graphics Processing Unit}). Metodą najbardziej przyspieszającą obliczenia,
lecz wymagającą najdłuższego czasu projektowania i najbardziej kosztowną,
jest zastosowanie specjalizowanych układów ASIC (ang. \emph{Application-Specific 
Integrated Circuit}). Opcją pośrednią pomiędzy powyższymi dwoma rozwiązaniami
jest zastosowanie układów FPGA. To podejście umożliwia osiągnięcie znacznego
przyspieszenia wykonywania obliczeń i nie powoduje wielkiego wzrostu kosztów. Dodatkowo 
zastosowanie metody HLS ułatwia i minimalizuje czas tworzenia sprzętowej implementacji 
modelu ANN oraz wprowadzanie zmian w projekcie. 
